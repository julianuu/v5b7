\paragraph{Fourier restriction theory 2}
$f∈L^1(ℝ^d)$ implies $\wh f$ is uniformly continuous (so, $\wh f$ can be restricted to any set)

$f∈L^2(ℝ^d)$ iff $\wh f∈L^2$, (therefore $\wh f$ cannot be restricted to a set of Lebesgue measure 0)

\paragraph{Question 1} What happens for intermediate $p∈(1,2)$?

Let $M⊂ℝ^d$ be smooth compact hypersurface equipped with $\mdμ=ψ\mdσ,\ ψ∈C^∞_0$ and $\mdσ$ surface measure on $M$. Given $1<p<2$, for which exponents $q$ does
\[(∫_M|\wh f(ξ)|^q\mdμ_ξ)^{\f1q}\lesssim\|f\|_{L^p(ℝ^d)}\]
hold? A complete answer for $q=2$ is given by Tomas-Stein: $M$ compact hypersurface whose Gauss curvature does note vanish on $\supp(M)$. Then the restriction inequality holds provided $q=2$ and $1\leq p\leq\f{2d+2}{d+3}$. Note, that this is the dual version.

\paragraph{Question 2} What happens for $q<2$? Dimensional analysis implies 
\[1\leq p<\f{2d}{d+1}\]
and Knapp-Type examples
\[q\leq(\f{d-1}{d+1})p'\]
The restriction conjecture states that these conditions are also sufficient.


\begin{proof}[End of proof of Tomas-Stein]
	Back to $(\se^{d-1},σ)$. Strategy:
	\[\|f*\whσ\|_{L^q(ℝ^d)}\lesssim\|f\|_{L^{q'}(ℝ^d)}\]
	if $q>\f{2d+2}{d-1}$.

	Why is 
	\[\|\wh k_j\|_{L^∞}\lesssim 2^j\]
	$k_j=ϕ_{2^{-j}}\whσ$ implies $\wh k_j=ψ^{2^{-j}}*σ$ where $ψ^ε(x)=ε^{-d}ψ(ε^{-1}x)$ and $ψ=\whϕ∈S$. Now
	\[|\wh k_j(ξ)|\lesssim_N2^{jd}∫_{\se^{d-1}}(1+2^j|ξ-η|)^{-N}\mdσ(η)\qquad∀N∈ℕ\]
	Now let $D(x,r)=B(x,r)∩\se^{d-1}$. Then $σ(D(x,r))\lesssim r^{d-1}$. So
	\begin{align*}
		|\wh k_j(ξ)|&\lesssim2^{jd}∫_{D(ξ,2^{-j})}(1+2^j|ξ-η|)^{-N}\mdσ(η)\\
		      &\qquad+\sum_{k\geq 0}∫_{D(ξ,2^{k+1-j})\sm D(ξ,2^{k-j})}(1+2^j|ξ-η|)^{-N}\mdσ(η)
	\end{align*}
	The first one will dominate because of the rapid decay of $ψ$.
	\[\lesssim2^{jd}[\ub{σ(D(ξ,2^{;j}))}_{\sim2^{-j(d-1)}}+\sum_{k\geq 0}2^{-Nk}\ub{σ(D(ξ,2^{k+1-j})\sm D(ξ,2^{k-j}))}_{\sim2^{(d-1)(k-j)}}]\lesssim2^j\]
	just choose $N=d$ such that the sum becomes a geometric series.
\end{proof}
\begin{rem}
	The $L^2→L^2$ bound in the previous argument was based only on dimensionality considerations. Therefore there should be an $L^2$ bound for $\wh{fν}$ valid under very general conditions.
\end{rem}
\begin{theo}
	Let $ν$ be a positive finite measure where
	\[ν(D(x,r))\leq cr^α\]
	Then
	\[\|\wh{fν}\|_{L^2(D(0,R))}\leq cR^{\f{d-α}2}\|f\|_{L^2(\mdν)}\]
\end{theo}
The proof relies on Schur's test: $(x,μ),\ (Y,ν)$ measure spaces, $K(x,y)$ measurable on $X\times Y$. If for all $y$ respectively $x$
\[∫_X|K(x,y)|\mdμ(x)\leq A,\qquad∫_Y|K(x,y)|\mdν(y)\leq B,\]
then for
\[(T_Kf)(x)=∫_YK(x,y)f(y)\mdν(y)\] we have
\[\|T_K\|_{L^2→L^2}\leq\sqrt{AB}.\]
\begin{proof}[Proof of the theorem]
	Let $ϕ∈S(ℝ^d)$ be radial such that $ϕ\geq 1$ on the unit disc $\whϕ$ has compact support. $ϕ_ε(x):=ϕ(εx)$. Then
	\[\|\wh{fν}\|_{L^2(D(0,R))}\leq\|ϕ_{R^{-1}}(x)\wh{fν}(-x)\|_{L^2_x(ℝ^d)}=\|\wh{ϕ_{R^{-1}}}*(fν)\|_{L^2(ℝ^d)}=\|∫_YR^d\whϕ(R(x-y))f(y)\mdν(y)\|_{L^2}\]
	We have the estimates
	\[∫_{ℝ^d}R^d\whϕ(R(x-y))|\md x=\|ϕ\|_{L^1}<∞\]
	by change of variables.
	\[∫_YR^d|\whϕ(R(x-y))|\mdν(y)\lesssim R^{d-α}\]
	by the hypothesis of $ν$ and compact support of $\whϕ$. Now apply Schur.
\end{proof}
\paragraph{Proof of the restriction conjecture in $d=2$}(Zygmund '70, Fefferman '72)


\begin{theo}
	Let $γ:I→ℝ^2$ be a smooth curve with $γ'\neq0$ and $γ''\neq 0$ on some finite interval $I$. Let $4<q\leq∞$ and $3p'\leq q$. Then $∀φ∈L^p(I)$
	\[\|∫_Iφ(t)e^{iγ(t)ξ}\md t\|_{L^q_ξ(ℝ^2)}\lesssim_{p,q}\|φ\|_{L^p(I)}\]
\end{theo}
Note, that the unit circle is a special case.
\begin{proof}
	Step 1: 'Even integer' trick.
	\[\|∫_Iφ(t)e^{iγ(t)ξ}\md t\|_{L^q_ξ(ℝ^2)}^2=∫_I∫_Iφ(t)\overline{φ(s)}e^{i(γ(t)-γ(s))ξ}\md t\md s\|_{L^{\f q2}(ℝ^2)}\]

	Step 2: Change variables $I\times I→U⊂ℝ^2,\ (t,s)↦γ(t)-γ(s)=x$. Choose $I$ small enough such that the change of variables is invertible with Jacobian $J=\det(\f{∂(t,s)}{∂x})$ satisfying $|J|\simeq|t-s|^{-1}$. Let $γ(t)=(x_1(t),x_2(t))$. Then
	\[|\f{∂x}{∂(t,s)}|=|
		\begin{pmatrix}
			-x_1'(s)&-x_2'(s)\\x_1'(t)&x_2'(t)
		\end{pmatrix}
		|
		=|x_1(s)x_2'(t)-x_2'(s)x_1'(t)|\simeq|t-s|
	\]
	since in case of $γ$ being the unit circle parametrized by arclength, this is equal to
	\[|\cos s\sin t-\sin s\cos t|=|\sin(s-t)|\sim|s-t|.\]
	Lets assume $γ$ is already parametrized by arclength. To compare that situation with the unit circle, define
	\[θ(s)=∫_0^sk(t)\md t,\]
	where $k$ is the (name?) curvature of $γ$. Then
	\[γ'(s)=(\cosθ(s),\sinθ(s))\]
	and due to
	\[(\min k)|s-t|\lesssim\|θ(s)-θ(t)|\leq\|k\|_{L^∞}|s-t|\]
	we can apply the above estimate.

	Step 3: Hausdorff-Young

	The integral now becomes
	\[\|∫_Ue^{ixξ}F(x)\md x\|_{L^{\f q2}(ℝ^2)}\leq\|F\|_{L^r(ℝ^2)}\]
	where
	\[F(x):=φ(t)\overline{φ(s)}|J|\]
	provided $\f q2=r'\geq 2$.

	Step 4: Fractional integration (Hölder-Littlewood-Sobolev)

	We treat the following as an integral of $|φ|^r$ times a second function.
	\[\|F\|_{L^r(ℝ^2)}\simeq(∫_I∫_I\f{|φ(t)|^r|φ(s)|^r}{|t-s|^{r-1}}\md t\md s)^{\f 1r}\leq c\|φ\|_{L^p(I)}^2\]
	provided $1<r<2$ and $1+\f1{(p/r)'}\geq r-1+\f1{p/r}$. The first one is fine by assumption. The second is equivalent to $3p'\leq q$.
\end{proof}
