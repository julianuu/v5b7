today:
\[(A_tf)(x)=∫_{|y|=1}f(x-ty)\mdσ(y)=(f*σ_t)(x)\]
which is the average of $f$ over the sphere of radius $t$ centered at $x$,
\[∫_{|x|=t}g(x)\mdσ_t(x):=∫_{|x|=1}g(tx)\mdσ(x).\]
This definition works fine provided $g$ is continuous.

Question: We would like to know for general $f$ whether $(A_tf)(x)→f(x)$ ($x$-a.e.) as $t→0$. Recall that we already did this for balls instead of spheres.

Is $t↦(A_tf)(x)$ continuous (for any $x$)? Is $\sup_{t_1<t<t_2}|A_tf(x)|$ measurable in $x$? This actually has to be discussed before we can think about the first question.

A priori estimate for the spherical maximal averages:
\begin{theo} Let $f∈C^0_0(ℝ^d),\ d\geq 4$. Then
	\[\|\sup_{t>0}|(A_tf)(x)|\|_{L^2_x(ℝ^d)}\lesssim\|f\|_{L^2(ℝ^d)}.\]
\end{theo}
Note that this can be improved up to $d\geq 2$, but fails for $d=1$.

Key: Let $f∈C^1_0(ℝ^d)$ and
\[(Sf)(x)=(∫_0^∞|\f{∂A_tf}{∂t}(x)|^2t\md t)^{\f12}.\]
This also holds for $C^0_0$ by density.
\begin{lem}
	\[\sup_{t>0}|(A_tf)(x)|\leq(Mf)(x)+c(Sf)(x)\]
	where $Mf$ is the standard Hardy-Littlewood maximal function over centered balls.
\end{lem}%1
\begin{proof}[Proof of lemma]	
	\[(A_tf)(x)=t^{-d}∫_0^t\f∂{∂s}[s^d(A_sf)(x)]\md s=I+\mathit{II},\]
	using that the integrand equals
	\[ds^{d-1}(A_sf)(x)+s^d\f{∂(A_s)f}{∂s}(x).\]
	\begin{align*}
		I&=dt^{-d}∫_0^ts^{d-1}(∫_{|y|=1}f(x-sy)\mdσ(y))\md s=dt^{-d}∫_{B(x,t)=\{y:|x-y|\leq t\}}f(y)\md y\\
		 &=\f1{\f{t^d}d}∫_{B(x,t)}f\leq\sup_{t>0}\dashint_{B(x,t)}f=(Mf)(x)
	\end{align*}
	due to our choice of normalization $|B(x,t)|=∫_0^t\ub{ω_{d-1}}_1s^{d-1}\md s=\f{t^d}d$.
	\begin{align*}
		\mathit{II}&=t^{-d}∫_0^ts^{d-\f12}\f{∂(A_sf)}{∂s}(x)s^{\f12}\md s\\
		&\leq(∫_0^∞|\f{∂(A_sf)(x)}{∂s}|^2s\md s)^{\f12}\cdot t^{-d}(∫_0^ts^{2d-1}\md s)^{\f12}\lesssim_d(Sf)(x)
	\end{align*}
\end{proof}
\begin{lem}
	If $d\geq 4$ then 
	\[\|Sf\|_{L^2(ℝ^d)}\leq A\|f\|_{L^2(ℝ^d)}	
	\]
\end{lem}
\begin{proof}
	$A_sf=f*σ_s\therefore$
	\[\hat{(A_sf)}(ξ)=\hat f(ξ)\hatσ_s(ξ)=\hat f(ξ)\hatσ(sξ)\]
	since
	\[\hatσ_s(ξ)=∫_{|x|=1}e^{2πixξ}\mdσ_s(x)=∫_{|x|=1}e^{2πisxξ}\mdσ(x)=\hatσ(sξ).\]
	It follows that
	\[\hat{\f{∂(A_sf)}{∂s}}(ξ)=\f{μ(sξ)}s\hat f(ξ)\]
	where 
	\[μ(ξ)=\sum_{j=1}^dξ_j\f{∂\hatσ}{∂ξ_j}(ξ)=\langleξ,∇\hatσ(ξ)\rangle\]
	because we differentiate with respect to a variable independent from the Fourier transform.

	Key: \[|μ(ξ)|\leq A\min\{|ξ|,|ξ|^{-\f{d-3}2}\}\]
	\begin{proof}
		\[\f{∂\hatσ}{∂ξ_j}(ξ)=2π∫_{|x|=1}x_je^{2πixξ}\mdσ(x)\]
		Since $x_j∈[-1,1]$ we get
		\[|\f{∂\hatσ}{∂ξ_j}(ξ)|\leq A(1+|ξ|)^{-\f{d-1}2}\]
		The statement follows from this and Cauchy-Schwarz applied to the definition of $μ(ξ)$.
	\end{proof}
	By Plancherel we get
	\[∫_{ℝ^d}|\f{∂(A_sf)}{∂s}(x)|^2\md x=∫_{ℝ^d}|\hat f(ξ)|^2\f{|μ(sξ)|^2}{s^2}\mdξ\]
	Now we multiply by $s$ and integrate with respect to $s$ and get
	\[∫_{ℝ^d}|Sf(x)|^2\md x=∫_{ℝ^d}|\hat f(ξ)|^2\ub{(∫_0^∞\f{|μ(sξ)|^2}s\md s)}_{*}\md ξ\]
	It is enough to show, that $*$ is bounded with respect to $ξ$.
	\[∫_0^∞\f{|μ(sξ)|^2}s\md s=∫_0^{\f1{|ξ|}}+∫_{\f1{|ξ|}}^∞\leq A(|ξ|^2∫_0^{\f1{|ξ|}}s^2\f{\md s}s+|ξ|^{-(d-3)}∫_{\f1{|ξ|}}^∞s^{-(d-3)}\f{\md s}s)\leq C\]
	where $C$ is independent of $ξ$. The whole thing is true iff $-(d-3)-1<-1,\ d\leq 4$, since this takes care of the second summand and the first summand is no problem.
\end{proof}
Now the theorem follows from the Lemmas and the $L^2$ boundedness of the maximal function.

Now look at $d=1$. Then the theorem fails because
\[A_tf(x)=\f12(f(x+t)+f(x-t))\]
and we can take as a counteraxample a function nonnegative which blows up close to $0$ but is in $L^p$ for every $p<∞$ such that $\sup_{t>0}(A_tf)(x)=∞$ everywhere.

He said something with the wave equation and its smoothing properties for $1\leq d\leq 3$.

$d\geq 2$: The maximal operator $f↦\sup_{t>0}(A_t(f)|$ is bounded on $L^p(ℝ^d)$ if $p>\f d{d-1}$.
\begin{itemize}
	\item For $d\geq 3$, this is in Stein's Chapter XI paragraph 3.  Ideas: rotational curvature, FIOs, dyadic decomposition, almost orthogonality
	\item For $d=2$, this is a theorem of Bourgan (1986) with alternative proofs by Sogge (1991): Cinematic curvature. Mockenhaupt-Seeger-Sogge (1993): Local smoothing for wave equation. $d=2$ is also in Stein's Chapter XI paragraph 4D
	\item No such result holds in $L^p(ℝ^d)$ if $p\leq\f d{d-1}$: Let
		\[f(y)=\f{|y|^{1-d}}{\log\f1{|y|}}1_{\{|y|\leq\f12\}}(y)\]
		Then $f∈L^p$ if $p\leq\f d{d-1}$: First, forget about the log for a moment. It is only there to take care of the endpoint.
		\[\|f\|_{L^p(ℝ^d)}^p\simeq∫_0^{\f12}\f{r^{(1-d)p}}{(\log\f1r)^p}r^{d-1}\md r=∫_0^{\f12}r^{(1-d)(p-1)}\md r\]
		\[(1-d)(p-1)>-1\quad(d-1)(p-1)<1\quad p<\f1{d-1}+1=\f d{d-1}\]
		For any $x$, the quantity $(A_tf)(x)$ is unbounded when $f\sim|x|$:
		\[\therefore\sup_t(A_tf)(x)=∞\quad\tx{everywhere}\]
\end{itemize}
Next: Averages with respect to a fixed curve:
\[(Mf)(x)=\sup_{h>0}\f1{2h}|∫_{-h}^hf(x-(t,t^2))\md t|\]
