\paragraph{Oscillatory integrals in harmonic analysis}
Stein (VIII,IX), Stein-Shakarchi (Chapter 8), Sogge
\begin{itemize}
	\item averaging operators
	\item restriction theory
	\item Bochner-Riesz summability
\end{itemize}
\subparagraph{Motivation (in $ℝ^3$)}
\[(Af)(x)=\dashint_{S^2}f(x-y)\mdσ_y=\f1{4π*σ}(x)\]
$σ$ surface measure in $S^2$.

Smoothing properties:
\begin{equation}
	\|\f∂{∂x_j}A(f)\|_{L^2(ℝ^3)}\lesssim\|f\|_{L^2(ℝ^3)}\quad(j=1,2,3)
	\label{eq:smoothprop}
\end{equation}
$f∈L^2⇒$
\[\|f*σ\|_{L^2}\leq\|f\|_{L^2}\ub{|σ|(ℝ^3)}_{<∞}\] (use Minkowski integral inequality) $\therefore f*σ∈L^2$ if $f∈L^2$.

Idea $\widehat{(f*σ)}=\hat f\hatσ$
\[\widehat{(\f∂{∂x_j}A(f))}(ξ)=ξ_j\hat f(ξ)\hatσ(ξ),\]
so we should study $\hatσ$.

without loss of generality $ξ=(0,0,|ξ|)$ because the integral is spherically symmetric.
\[\hatσ(ξ)=∫_{S^2}e^{-2πiωξ}\mdσ_ω=2π∫_0^πe^{-2πi|ξ|\cosθ}\sinθ\md=2π∫_{-1}^1e^{2πi|ξ|t}\md t=\f{2\sin(2π|ξ|)}{|ξ|}\]
\[t=\cosθ\quad\md t=\sinθ\mdθ\]
\[\therefore|\hatσ(ξ)|\lesssim(1+|ξ|)^{-1}\]
\eqref{eq:smoothprop} follows from this and Plancherel.

This also generalizes to higher dimensions. There we get a factor $(1-t^2)^{\f{d-3}2}$ in the last integral.

\paragraph{Oscillatory integrals}
\[I(λ)=∫_{ℝ^d}e^{iλϕ(x)}ψ(x)\md x\]
where $λ∈ℝ$ is the oscillatory parameter, $ϕ∈C^∞(ℝ^d,ℝ)$ the phase and $ψ(x)∈ℂ$ the amplitude.

Q.: How does $I(λ)$ behave for large $|λ|$? General principle: Main contribution comes from the critical points of the phase, $x_0:\ ∇ϕ(x_0)=0$.

\paragraph{Principle of non-stationary phase}
$ϕ∈C^∞,\ ψ∈C^∞_0:|∇ϕ(x)|> 0\ (∀x∈\suppψ)$. Then for any $N∈ℕ$
\[|I(λ)|\leq c_N|λ|^{-N}\]
\begin{proof} Integration by parts (in a more complicated version)
\end{proof}
$d=1$:
\[I_1(λ)=∫_a^be^{iλϕ(x)}\md x\]
\[0<a<b<∞\quadψ(x)=χ_{[a,b]}(x)\quad\tx{which is rough!}\]
This means we will not get such a fast decay
\begin{lem}[van der Corput (I)]
	$ϕ∈C^2,\ ϕ'$ monotonic, $|ϕ'(x)|\geq 1\ (∀x∈[a,b])$. Then
	\[|I_1(λ)|\leq\f3{|λ|}\quad(∀λ>0)\]
\end{lem}
\begin{rem}
	\begin{enumerate}
		\item $3$ is neither important nor sharp; independence of $a,b,ϕ$ is the key!
		\item Order of decrease in $λ$ is sharp ($ϕ()=x\therefore I_1(λ)=\f{e^{iλb}-e^{iλa}}{iλ}$)
		\item monotonicity of $ϕ'$ is essential
	\end{enumerate}
\end{rem}
\begin{proof} Integrate by parts(…)
\end{proof}
What if critical points are present? ($d=1$)

$x_0:ϕ'(x_0)=0$ (critical point) and $ϕ''(x_0)\neq 0$ (non degenerate), e.g.\ $ϕ(x)=x^2,\ x_0=0$. In this case
\[∫_ℝe^{iλx^2}ψ(x)\md x=c_0λ^{\f{-1}2}+\ord(|λ|^{-\f32})=\sum_{k=0}^Na_xλ^{-\f12-k}+\ord(|λ|^{-\f32-N})\quad(∀N,\ λ→∞)\]
\begin{lem}[van der Corput (II)]
	$ϕ∈C^2[a,b],\ |ϕ''(x)|\geq 1\ (∀x∈[a,b])$. Then
	\[|I_1(λ)|\leq\f8{λ^{\f12}}\quad(∀λ>0)\]
\end{lem}
\begin{rem} More generally: $\ord(|λ|^{\f1k})$ if $|ϕ^{(k)}|\geq 1$.
\end{rem}
\begin{proof} Integration by parts not needed. Instead split up region in small area around critical point with properly chosen size, and rest, and then use results from above.
	%figure here?
\end{proof}
\begin{cor} Same assumptions as van der Corput (II). $ψ∈C^1[a,b]$.
	\[|∫e^{iλϕ(x)}ψ(x)\md x|\leq c_ψλ^{\f{-1}2}\]
\end{cor}
Application: Asymptotics of Bessel functions
\[J_m(r)=\f1{2π}∫_0^{2π}e^{ir\sin x}e^{-imx}\md x\quadϕ(x)=\sin x,\quad ψ(x)=e^{-imx}\quad(m∈ℤ)\]

\begin{cor}
	\[|J_m(r)|\leq cr^{-\f12}\quad r→∞\]
\end{cor}
Recall: Averaging operator in $ℝ^d\ (d>1)$ is 
\[(Af)(x)=(f*σ)(x)\quadσ\tx{ surface measure on $S^{d-1}$}\]
\begin{theo}
	$f⇒A(f)$ is bounded from $L^2(ℝ^d)$ to $L^2_k(ℝ^d)$ with $=\f{d-1}2$.
\end{theo}
\begin{proof}
	\[\hatσ(ξ)=2π|ξ|^{-\f d2+1}\ub{J_{\f d2-1}(2π|ξ|)}_{=\ord(|ξ|^{-\f12}),\ |ξ|→∞}\]
	\[\therefore|\hatσ(ξ)|=\ord(|ξ|^{-\f{d-1}2})\quad|ξ|→∞\]
\end{proof}	

"What is van der Corput's lemma in higher dimension?" (Carbery-Wright, 2000)

\[I(λ)=∫_{ℝ^d}e^{iλϕ(x)}ψ(x)\md x\]
($ϕ$ smooth, $ψ$ smooth, compactly supported)

nondegeneracy hypothesis
\[\det(∇^2ϕ)(x)\neq 0\quad∀x∈\supp(ψ)\]

\begin{theo} Under above assumptions
	\[|I(λ)|=\ord(|λ|^{-\f12})\quadλ→∞\]
\end{theo}	
\begin{rem}
	\begin{enumerate}
		\item Decay rate is sharp
		\item Proof uses $TT^*$ method: $|I(λ)|^2=I(λ)\overline{I(λ)}$
		\item variant: $\rk(∇^2ϕ)\geq m$ for some $0<m\leq d$ on $\supp(ψ)$. Then
			\[|I(λ)|=\ord(|λ|^{-\f m2})\]
	\end{enumerate}
\end{rem}
\paragraph{Application:} Fourier transform of surface-carried measures

Recall: $(S^{d-1},σ)$
\[|\hatσ(ξ)|\lessim(1+|ξ|)^{-f{d-1}2}\]
(not a Bessel coincidence)

(local) $C^∞$-hypersurface $M$. After translation and rotation $x_0=0$, $T_{x_0}M=\{x_d=0\}$. $M$ can be represented as
\[M=\{(x',x_d)∈B⊂ℝ^d:x_d=φ(x')\}\]
Can arrange $φ(0)=0=(∇_{x'}φ)(x')|_{x'=0}$.
\[φ(x')=\f12\sum_{k,j=1}^{d-1}\ub{\f{∂^2φ}{∂x_k∂x_j}}_{(a_{jk})}x_kx_j+\ord(|x'|^3)=\f12\sum_{j=0}^{d-1}kjx_j^2+\ord(|x'|^3)\]
$(a_{jk})$ $(d-1)\times(d-1)$ $ℝ$-valued symmetric matrix $\therefore$ diagonizable. $k_j$ principal curvaturos of $M$ at $x_0$. $k:=\prod_{j=1}^{d-1}k_j$ is the Gaussian curvature of $M$ at $x_0$ ($k=\det(∇^2φ)$)

E.g.\ 
\begin{enumerate}
	\item $S^{d-1}⊂ℝ^d$. $k_j=1\ (∀j)\therefore k=1$
	\item $\{x_3=\ub{x_1^2-x_2^2}_{φ(x_1,x_2)}\}⊂ℝ^3$, $\f12∇^2φ(x)=
	\begin{pmatrix}1&0\\0&-1\end{pmatrix}$
	\item $\{x_1^2=|x'|^2:x\neq 0\},\ x'∈ℝ^{d-1}$. $d-2$ identical nonvanishing principal curvatures $x_d^{-2}$ +1 vanishing principal curvature.
\end{enumerate}
\paragraph{surface measure $σ$}
\[∫_Mf\mdσ=∫_{ℝ^{d-1}}f(x',φ(x'))\ub{\sqrt{1+|∇_{x'}φ(x')|^2}\md x'}_{\mdσ\tx{ in our coordinate sys.}}\]
\[\mdμ=ψ\mdσ,\quadψ∈C^∞_0(M,σ)\]
is a surface carried measure.
\[\hatμ(ξ)=∫_Me^{-2πixξ}\mdμ(x)=∫_Me^{-2πixξ}ψ(x)\mdσ_x\]
is bounded on $ℝ^d$ because $|μ|(ℝ^d)<∞$.
\begin{theo} Hypersurface $M⊂ℝ^d$ with nonvanishing Gaussion curvature at each point of $\supp(ψ)$. Then
	\[|\hatμ(ξ)|=\ord(|ξ|^{-\f{d-1}2}),\quad|ξ|→∞\]
\end{theo}
\begin{cor} If $M$ has at last $m$ non vanishing principal curvatures (at each point of $\supp(ψ)$), then
	\[|\hatμ(ξ))|=\ord(|ξ|^{-\f m2}),\quad |ξ|→∞\]
\end{cor}
