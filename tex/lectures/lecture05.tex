\paragraph{Brunn-Minkowski ineq. ($ℝ^d$)}
$A,b⊂ℝ^d$ measurable. $A+B=\{a+b:a∈A,,b∈B\}$. $λA=\{λa:a∈A\}$ ($λ>0$).

Q.: Can $|A+B|$ be controlled in terms of $|A|,|B|$? No! There exist sets $A,B$ $|A|=|B|=0$ with $|A+B|>0$. Example $[0,1]\times[0,1]$. Another example $A=B=C⊂[0,1]$ Cantor set. Then $A+B=[0,2]$.

Q.: Can $|A+B|^α\geq c_α(|A|^α+|B|^α)$ hold? (for some $α>0$ with $c_α<∞‚$ indep of $A,B$) Best possible $c_α=1$.

What about $α$? Convex sets play a role. $A=$ convex, $B=λA$. $|B|=|λA|=λ^d|A|$. $|A+B|=|A+λA|=|(1+λ)A|=(1+λ)^d|A|$ because $A$ is convex.

($λ_1A+λ_2A=(λ_1+λ_2)A$ iff $A$ is convex.)

$|A+B|^α\geq |A|^α+|B|^α$ iff $(1+λ)^{dα}\geq 1+λ^{dα}$ $⇒α\geq\frac1d$.

$(a+b)^γ\geq a^γ+b^γ∀a,b\geq 0,\ γ\geq 1$.

Candidate inequality: 
\begin{equation}
	|A+B|^{\frac1d}\geq|A|^{\frac1d}+|B|^{\frac1d}
	\tag{\text{BM}}
	\label{eq:bm}
\end{equation}

$A,B$ measurable $\not\implies A+B$ measurable ($6=$'not'; weird bug in \texttt{unicode-math}). Take $[0,1]\times\text{nonmeasurable}$.

\begin{enumerate}
	\item $A,B$ closed $⇒$ $A+B$ measurable
	\item $A,B$ compact $⇒$ $A+B$ compact
	\item $A,B$ open $⇒$ $A+B$ open
\end{enumerate}
\begin{theo} \eqref{eq:bm} holds if $A,B,A+B$ measurable.
\end{theo}
\begin{proof}
	\begin{enumerate}
		\item $A,B$ rectangles with sidelengths $\{a_j\}_{j=1}^∞,\ \{b_j\}_{j=1}^∞$ which are parallel to the axes\label{it:rect}
		\item $A,B$ unions of finitely many such rectangles with disjoint interiors.\label{it:manyrect}
		\item $A,B$ open sets of finite measure\label{it:open}
		\item $A,B$ compact\label{it:cpt}
		\item $A,B,A+B$ measurable.\label{it:meas}
	\end{enumerate}
	\begin{itemize}
		\item[\ref{it:rect}] \eqref{eq:bm} becomes \[\prod_{j=1}^d(a_j+b_j)^{\frac1d}\geq\prod_{j=1}^da_j^{\frac1d}+\prod_{j=1}^db_j^{\frac1d}\]
			$a_j\toλ_la_j,\ b_j\toλ_jb_j$. Both sides are multiplied by $(λ_1λ_2…λ_d)^{\frac1d}$: wglog can assume $a_j+b_j=1\ ∀j$ (Choose $λ_j=a_j+b_j$)

			geometric mean $\leq$ arithmetic mean:\[\prod_{j=1}^da_j^{\frac1d}\leq\frac1d\sum_{j=1}^da_j\]
			\[\prod_{j=1}^db_j^{\frac1d}\leq\frac1d\sum_{j=1}^db_j\]
			\[\prod a_j^{\frac1d}+\prod b_j^{\frac1d}\leq\frac1d\sum_{j=1}^d(a_j+b_j)=1\]
		\item[\ref{it:manyrect}] Induction on $n=$ number of rectangles in $A$ and $B$. Choose pair of disjoint rectangles $R_1,R_2$ in $A$. Can rotate s.t. $R_1$ and $R_2$ are separated by hyperplane $\{x_j=0\}$. $R_1$ lies in $A_+=A∩\{x_j\geq 0\},\ A_I=A∩\{x_j\leq 0\}$.

			Rem.: Both $A_+,A_-$ contain at least one less rectangle than $A$, $A=A_+⊂A_-$ and $A_+∩A_-$ has measure zero.

			Now: translate $B$ s.t. $B_-$ and $B_+$ satisfy \[\frac{|B_\pm|}{|B|}=\frac{|A_\pm|}{|A|}\]
			$(A_++B_+)∪(A_-+B_-)⊂A+B$ Number of rectangles in $A_+$ and $B_+$, number of rectangles in $A_-$ and $B_-$ is $<n$.

			\begin{align*}
				|A+B|&\geq|A_++B_+|+|A_-+B_-|\geq(|A_+|^{\frac1d}+|B_+|^{\frac1d})^d+(|A_-|^{\frac1d}+|B_-|^{\frac1d})^d\\
				     &=(|A_+|(1+(\frac{|B_+|}{|A_+|})^{\frac1d})^d+|A_-|(1+(\frac{|B_-|}{|A_-|})^{\frac1d})^d=(|A_+|+|A_-|)(1+(\frac{|B|}{|A|})^{\frac1d})^d\\
				     &=(|A|^{\frac1d}+|B|^{\frac1d})^d.
			\end{align*}
		\item[\ref{it:open}] Open sets of finite measure $A,B$. $∀ε>0∃A_ε,B_ε$ finet unions of parallel rectangles with disjoint interiors. $A_ε⊂A,B_α⊂B$, $|A|\leq|A_ε|+ε,\ |B|\leq|B_ε|+ε$.

			$|A+B|\geq |A_ε+B_ε|\geq(|A_ε|^{\frac1d}+|B_ε|^{\frac1d})^d\geq((|A|-ε)^{\frac1d}+(|B|-ε)^{\frac1d})^d$. Let $ε→0^+$, done.
		\item[\ref{it:cpt}] $A,B$ compact. Let $A^ε=\{x:d(x,A)<ε\}$. $A+B⊂A^ε+B^ε⊂(A+B)^{2ε}$. Then use that for compact sets $|C^ε|\to|C|$.
		\item[\ref{it:meas}] $A,B,A+B$ measurable: use inner regularity of Lebesque measure.
	\end{itemize}
\end{proof}

\begin{rem} $A,B$ open sets of finite positive measure. Equality in (BM) iff $A,B$ convex and similar. $∃δ>0∃h∈ℝ^d:A=δB+h$ ($A$ convex iff $λ_iA+λ_2A=(λ_1+λ_2)A$)
\end{rem}
\paragraph{Consequences for isoperimetric inequality}
$A⊂ℝ^d$ bounded open with smooth boundary. ($∂A$, $B⊂ℝ^d$ ball $|B|=|A|$)
\[|∂A|=\lim_{ε→0^+}\frac{|A+εB|-|A|}ε\]
Isoper ineq.: $|∂A|\geq|∂B|$.
\begin{proof}
	\[\frac{|A+εB|-|A|}ε\geq\frac{(|A|^{\frac1d}+|εB|^{\frac1d})^d-|A|}ε=\frac{(1+ε)^d-1}ε|B|→d|B|=|∂B|\]
	for $ε→0$.
\end{proof}
Better: $A⊂ℝ^d$ has finite perimeter ($\iff 1_A∈\BV(U),\ U⊂ℝ^d$ bdd open)
\[\frac{\hm^{d-1}(∂A)}{|A|^{\frac{d-1}d}}\geq\frac{\hm^{d-1}(S^{d-1})}{|B^d(0,1)|^{\frac{d-1}d}}\]

\paragraph{Hausdorff measure} 
Q: How does a set replicate under scaling? $E→ nE=E_1∪…∪E_m$ disjoint congruent copies of $E$. Examples: line $m=n^1$, square $m=n^2$, cube $m=n^3$, Cantor set $3C=C_1∪C_2$ $2=3^α\iffα=\frac{\log3}{\log2}$

$\#(ε)=$least $\#$ of segments that arise from such poygonal lines. $Γ$ rectifiable iff $\#(ε)\simε^{-1}$ as $ε→0^+$. If $\#(ε)\simε^{-α}$ ($α>1$) In this case, say "$Γ$ has dim $α$". Snowflake has $α=\frac{\log 4}{\log3}>1$.

Upshot: $E$ $α>1$. $m_α(E)=$ $α$-dimensional mass of $E$ among sets of "dimension" $α$.
\begin{itemize}
	\item $α>\dim(E)⇒m_α(E)=0$
	\item $α<\dim(E)⇒m_α(E)=∞$
	\item $α=\dim(E)$ interesting
\end{itemize}
