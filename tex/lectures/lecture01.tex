\paragraph{Q1}
\begin{equation*}
	\left.
		\begin{matrix}
			f:[a,b]→ℝ\text{ integrable}\\
			F(x)=∫_a^xf(t)\md t
		\end{matrix}
	\right]\underset?⇒ F\text{ diff. (a.e. $x$)},\ F'=f
\end{equation*}
\paragraph{Q2} Conditions of $F$ (on $[a,b]$) s.t. 
\begin{itemize}
	\item $F'(x)$ exists a.e.
	\item $F'$ integrable
	\item $∫_a^bF'(x)\md x=F(b)-F(a)$
\end{itemize} ?
\paragraph{Q1} Differentiation of the integral
\begin{equation*}
	\frac{F(x+h)-F(x)}h=\frac1h∫_x^{x+h}f(t)\md t=\frac1{|I|}∫f=\avg_If=\intbar_If
\end{equation*}
$I=(x,x+b)$, $|I|$ Lebesgue measure of $I$.

Q1 equivalent to averaging problem: Given $f∈L^1(ℝ^d)$, is it true, that
\begin{equation*}
	\lim_{|B|→0,\ x\in B}=\frac1{|B|}∫_Bf=f(x)\quad (\text{$x$-a.e.)?}
\end{equation*}
$B⊂ℝ^d$ open ball

\highl{Yes}, if $f$ continuous $∀ε∃δ|x-y|<δ⇒|f(x)-f(y)<ε$. $x∈B$
\begin{equation}
	|f(x)=\dashint|=|\dashint_B(f(y)-f(x))\md y|<ε
\end{equation}
provided $B$ is an open ball of radius $<\fracδ2$ containing $x$

\highl{Yes}, if $f$ is integrable (not so easy). Hardy, Littlewood (1D, rearrangements; later Wiener for $d>1$). $f∈L^1(ℝ^d)$
\begin{equation*}
	(Mf)(x):=\sup_{x∈B}\frac1{|B|}∫_B|f|
\end{equation*}
\highl{uncentered} HL maximal function
\begin{theo}
	Let $f$ be integrable on $ℝ^d$. Then
	\begin{enumerate}
		\item $Mf$ is measurable.
		\item $(Mf)(x)<∞$ a.e. $x$
		\item 
			\begin{equation} 
				\{x∈ℝ^d:(Mf)(x)>α\}|<\frac cα\|f\|_{L^1(ℝ^d)}\ (∀x>0)\label{eq:maxweak}.
			\end{equation} $c=c_d=3^d$, independent of $f,α$.
	\end{enumerate}
\end{theo}
$f\neq f∈L^1 ⇒Mf(x)\sim|x|^{-d}$ for large radius of $x$. So then $Mf\not\in L^1$.
\begin{equation*}
	M:
	\begin{matrix}
		L^1\not\to L^1\\
		L^1→L^{1,∞}
	\end{matrix}
\end{equation*}
\begin{proof}
	\begin{enumerate}
		\item easy $E_α=\{x∈ℝ^d:(Mf)(x)>α\}$ is open ($∀x>0$) (because $Mf$ is lowes semicontinuous)
		\item $|\{x∈ℝ^d: (Mf)(x)=∞\}|⊂|\{x∈ℝ^d:Mf(x)>α\}|$, take $α→∞$.
		\item follows from an elemantary version of \highl{Vitali covering}
	\end{enumerate}
\end{proof}%not end proof
\begin{lem}
	Let $B=\{B_1,B_2,\ldots, B_N\}$ be a finite collection of open balls on $ℝ^d$. Then there exists a disjoint subcollection $B_{i_1},B_{i_2},\ldots,B_{i_k}$ of $B$ such that
	\begin{equation*}
		|\bigcup_{j=1}^nB_j|\leq3^d\sum_{j=1}^k|B_{ij}|
	\end{equation*}
\end{lem}
\begin{proof}
	\begin{enumerate}
		\item $B_{i_1}=$ largest ball
		\item Delete $B_{i_1}$ and its neighbors
		\item $B_{i_2}=$ largest ball
		\item repeat\ldots
	\end{enumerate}
	\begin{itemize}
		\item Algorithm stops in at most $N$ steps
		\item output has desired properties:
			\begin{itemize}
				\item disjointness is clear
				\item size $B∩B'\neq∅$, $r_{B'}\leq r_B$. $B^*=$ ball with the same center as $B$ but 3 times the radius. $⇒B'⊂B^*$. $|B^*|=3^d|B|$
			\end{itemize}
	\end{itemize}
\end{proof}
\paragraph{Back to (iii):} Choose $α>0$, $E_α=\{x∈ℝ^d:(Mf)(x)>α\}$. Fr each \[x∈E_α∃B=B_x:=\frac1{|B_x|}∫_{B_x}|f(y)|\md y>α\]
equivalent \[|B_x|<α^{-1}∫_{B_x}|f(y)|\md y\]
Fix $K\ll E_α$ compact subset covered by $\bigcup_{x∈K}B_x$, $K⊂\bigcup_{l=1}]NB_l$

\begin{align*}
|K|\leq|\bigcup_{l=1}^NB_l|\underset{\text{Vitali}}{\leq}3^d\sum_{j=1}^k|B_{ij}|\leq\frac{3^d}α∈_{j=1}^k∫_{B_{i_j}}|f()|\md y=\frac{3^d}α∫_{\bigcup_{j=1}^kB_{i_j}}|f(y)|\md y\leq\frac{3^d}α\|f\|_{L^1(ℝ^d)}
\end{align*}
Since $K$ was choseen arbitrary (cpt.), it follows that \[E_α|\leq\frac{3^d}α\|f\|_{L^1}\]
Can interpolate between weak type $L^1$-inequality and $L^∞→L^∞$ (very easy).
\begin{cor}[Lebesque differentiation theorem]
Let $f∈L^1(ℝ^d)$ Then 
\begin{equation}
	\lim_{|B|,→0,x∈B}\dashint f=f(x)\quad \text{$x$-a.e.}\label{eq:lebeqdiff}
\end{equation}
\end{cor}
\begin{proof}
	\begin{equation*}
		E_α=\{x∈ℝ^d:\limsup_{|B|→0,x∈B}|\dashint_Bf-f(x)>2α\}
	\end{equation*}
	ETS $|E_α|=0\ ∀α>0$. Then $E=\bigcup_{n∈ℕ}E_{\frac1n}=0$ and \eqref{eq:lebeqdiff} holds on $E^\complement$.

	Fix $α>0$, given $ε>0$ choose $g∈C^0_0(ℝ^d)$ s.t. $\|f-g\|_{L^1}<ε$. Already seen \[\lim_{|B|⇒0,x∈B}\dashint g=g(x)\ ∀x\]
	\[\dashint_Bf-f(x)=\dashint_B(f-g)+\dashint_Bg-g(x)+g(x)-f(x)\]
	\begin{align*}
		F_α=\{x:M(f-r)(x)>α\}\\
		G_α=\{x:|f(x)-g(x)|>α\}
	\end{align*}
	$E_α⊂F_α∪G_α$ since $u_1,u_2>0,\ u_1+u_2>2α⇒u_1>α∨u_2>α$.
	\begin{align*}
		|G_α|\leq\frac1α\|f-g\|_{L^1}\quad\text{(Chebyshew)}\\
		|F_α|\leq\frac{c_d}α\|f-g\|_{L^1}\quad\text{(weak type)}\\
		|E_α|\leq|F_α|+|G_α|\leq(\frac{c_d}α+\frac1α)\|f-g\|_{L^1}\leq\frac{c_d'ε}α
	\end{align*}
Since $ε>0$ was arbitrary $|E_α|=0$.
\end{proof}
$h∈L^1⊂L^{1,∞}$ by Chebyshew: $∞>\|h\|_{l^1}=∫_{ℝ^d}|h(y)|\md y\geq∫_{h(y)\geqα}|h(y)|\md y\geq α|\{|h|>α\}|$.

Would have been enough to replace $L^1(ℝ^d)$ by $L^1_\loc$.

\paragraph{Sets}
$E⊂ℝ^d$ measurable, $x∈ℝ^d$ (not necc. in $E$)
$x$ is a point of Lebesque density of $E$ if \[\lim_{|B|→0,x∈B}\frac{|B∩E|}B=1\]
\begin{cor}
	Let $E⊂ℝ^d$ be measurable. Then
	\begin{enumerate}
		\item Almost every $x∈E$ is a point of Lebesque density of $E$.
		\item Almost every $x\not\in E$ is not a point of Lebesque density.
	\end{enumerate}
\end{cor}
\paragraph{Functions}
$f∈L^1_\loc(ℝ^d)$. \[Leb(f):=\{x∈ℝ^d:f(x)<∞\text{ and }\lim_{|B|→0,x∈B}\dashint_B|f(y)-f(x)|\md y=0\}\]
$f$ continuous at $\bar x⇒\bar x∈\Leb(f)⇒\dashint_Bf\underset{|B|→0,x∈B}→f(\bar x)$ (all the inverse implications are wrong)
\begin{cor}
	$f∈L^1_\loc(ℝ^d)⇒$ Almost every point belongs to $\Leb(f)$.
\end{cor}
(By checking the proof again? I see no other way)

These things also works with other sets that "shrink regularly to $x$ than balls". It gets worse however when one takes all parallel rectangles and even worse when arbitrarily oriented rectangles are allowed.
