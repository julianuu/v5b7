\paragraph{Fourier restriction theory}
Gilen $f:\se^{d-1}→ℂ$, constider the Fourier transform $\hat{Fσ}(ξ)=∫_{\se^{d-1}}f(x)e^{/2πixξ}\mdσ(x)$ for $ξ∈ℝ^d$ (temdered distribution which turs out to be a function)
\begin{itemize}
	\item $f$ smooth $⇒\hat{fσ}$ decays at $∞$, e.g.
		\[|\hat{fσ}nξ)|\leq c\|f\|_{C^2}(1+|υξ|)^{-\f{d-1}2}\]
		via stationary phase
	\item $f$ bounded then no pointwise decay holds in general. E.g.\ \[f_k(x)=e^{2πikx}\]
		Consider $ξ=k$ then $|\hat{f_kσ}(ξ)|=σ(\se^{d-1})\sim 1$, no decay here. Now take
		\[f=\sum_{j\leq 0}\f{f_{k_j}}{j^2}\]
		where $|k_j|→∞$ sufficiently fast, e.g.\ $k_j=j!$. Eay to check: $f$ is continuous but $|\hat{fσ}(ξ)|\leq c(1+|ξ|)^{-ε}$ does not hold for any $c<∞,\ ε>0$. (uniformly in $ξ$).
	\item Problem of distinguished origin disappears if we take $L^q$ norms

\end{itemize}
Restriction conjecture (Stein): rove that öf $f∈L^∞(\se^{d-1})$, then
\[\|\hat{fσ}\|_{L^1(ℝ^d)}\leq c_{q,d}\|f\|_{L^∞(\se^{d-1},σ)}\]
fore very $q>\f{2d}{d-1}$.
\begin{itemize}
	\item Range of exponents would besharp: take $f\equiv 1∈L^ ‚(\se^{d-1})$
		\[\|\hatσ\|_{L^q(ℝ^d)}^q=∫_{ℝ^d}|\hatσ(ξ)|^q\md ξ\lesssim∫_{ℝ^d}(1+|ξ|)r^{-\f{d-1}2q}\mdξ=_d\ub{∫_0^∞(1+r)^{-\f{d-1}2q}r^{d-1}\md r}_{<∞}\]
		iff $-\f{d-1}2q+(d-1)<-1$ iff $q>\f{2d}{d-1}$.

\end{itemize}
Corresponding problem for $Lw2$ densities was solved by Tomas-Stein inequality ($\sim 1975$). If $f∈L^2(se^{d-1})$, then
\[\|\hat{fσ}\|_{L^q(ℝ^d)}\lesssim_{q,d}\|f\|_{L^2(\se^{d-1})}\]
if $q\geq\f{2d+2}{d-1}$, and this range is the best possible. Note, that this was already proven last term, can be found in Stein/Shakarchi

\begin{itemize}
	\item Assumptions are fo the form $q>q_0$ or $g\geq q_0$. Why?
		\[\|\hat{fσ}\|_{L^∞ℝ^d)}=\sup_{ξ∈ℝ^d}|∫_{\se^{d-1}}f(x)e^{-2ηixξ}\md σ_x|\leq∫_{\se^{d-1}}|f(x)|\mdσ_x=\|f\|_{L^1(\se^{d-1})}\lesssim\|f\|_{L^2(\se^{d-1})}\]
		+ Riesz-Thorin because the sphere is compact
	\item $q\geq\f{2d+2}{d-1}$ is the best possible for $L^2$ densities: Hnapp counterexample.
		\[C_δ=\{x∈\se^{d-1}:1-xe_d\leqδ^2\}\] wher $e_d=(0,…,0,1)$. Since $|x-e_d|^2=2(1-xe_d)$, \[|x-e_d|\leq cδ⇒x∈C_δ⇒|x-e_d|\leq Cd\]
		for appropriate constants $0<c<C<∞$. Let $f=1_{C_δ}$. Right hand side is easy:
		\[\|f\|_{L^2(\se^{d-1})}=|C_δ|^{\f12}\cong_dδ^{\f{d-1}2}\]
		Left hand is trickier. Note: the upport of $fσ$ is contained in a cylindrical box $B_δ$ centered at $e_d$ with length $\simδ^2$ in the $_d$ irection and $\simδ$ in the $(d-1)$ orthogonal direction.
		\[\|\hat{fσ}\|_{L^q(ℝ^d)}\]
		Uncertainty principle: If a function is supported it a box, then its fourier transform is more or less constant on the dual box, which is the same but with inverse lengths. Idea: look at $\hat{fσ}$ on the dual box $B_δ^*$ (centered at $0$), u.e.\ suppose $|ξ_d|\leqc_1^{-1}δ^{-1}$ and $|ξ_j|\leq c_1^{-1}δ^{-1}$ if $j<d$, $c_1$ is a large constant to be chosen. If $ξ∈B_δ^*$, then
		\[|\hat{fσ}(ξ)|\f|∫_{C_δ}e^{2πixξ}\mdσ_x|=∫e^{2πi(x-e_d)ξ}\mdσ_x|\geq∫_{C_δ}\cos(2π(x-e_d)ξ)\mdσ_x|\] Conditions on $ξ$ imply thät 
		\[2π(x-e_d)ξ|\leq\fπ3\]
		if $C_1$ langle enough. Therefore
		\[|\hat{fσ}(ξ)|\geq\f12|C_d|\congδ^{d-1}\]
		How lanv,gle is $B_^*$?
		\[|B_δ^*|\congδ^{-2}(δ^{-1})^{d-1}=δ^{-(d+1)}\]
		Conclusion:
		\[\|\hat{fσ}\|_{L^1(ℝ^d)}\gegsim(∫_{B_δ^*}|\hat{fσ}(ξ)|^q\mdξ)^{\f1q}\geqsimδw{d-1}δ^{-\f{d+1}q}\]
		\[δ^{d-1}δ^{-\f{d+1}q}\leq\|\hat{fσ}\|_{L^q(ℝ^d)}\lesssim\|f\|_{L^2(\se^{d-1})}\congδ^{\f{d-1}2}\]
		\[\therefore d-1-\f{d+1}q\geq\f{d-1}2⇔\f{d-1}2\geq\f{d+1}q⇔q\geq{2d+2}{d-1}\]
\end{itemize}
So, if you violate both of the two obstructions, then the inequality should hold (which obstructions?)

Technical tool: convolution of Schwartz function with a (compacty supported) measure. $ϕ∈S(ℝ^d),\ μ∈M(ℝ^d)$ then
\[(φ*μ)(x)=∫φ(x-y)\mdμ(y)\]
Notation
\[\check μ=\hatμ(-\cdot)\]
\begin{lem}
	\begin{enumerate}
		\item \[\hat{\check ϕμ^∨}=\hatϕ*μ\]
		\item \[\hat{ϕμ}=φ*\hatμ\]
	\end{enumerate}
\end{lem}
\begin{proof}[Proof of (b), (a) is similar]
	Enough to shw $∀ψ∈S(ℝ^d)$
	\[∫_{ℝ^d}\hat{ϕμ}ψ\md x=∫_{ℝ^d}(\hatφ*\hatμ)ψ\md x\]

	\begin{align*}
		∫_{ℝ^d}\hat{ϕμ}ψ\md x=∫\hatψφ\md μ=∫\hat{\checkϕ*ψ}\mdμ=∫_{ℝ^d}(\checkφ*ψ)\hatμ\md x=∫_{ℝ^d}(\hatϕ*\hatμ)ψ\md x
	\end{align*}
	due to duality, Fourier inversion on $S$, duality and definition of $*$ + Fubini.
\end{proof}
\begin{lem} $f,g∈S,μ∈M(ℝ^d)$. Then
	\[∫\hat f\bar{\hat g}\md μ=∫_{ℝ^d}(\hatμ*\bar g)f\md x\]
\end{lem}
\begin{proof}
	\[∫\hat f\bar{\hat g}\md μ=∫_{ℝ^d}f\hat{\bar{\hat g}μ}\md x=∫f_{ℝ^d}f(\bar g*\hat μ)\md x\]
	by duality and lemma 1
\end{proof}
\begin{lem} $μ$ finite positive measure. Then the following are equivalent
	\begin{enumerate}
		\item $\|\hat{fμ}\|_{L^q}\leq c\|f\|_{L^2(⇔)}\quad∀f∈L^2(μ)$
		\item $\|\hat g\|_{L^2(⇔)}\leq c\|g\|_{L^{q'}}\quad∀g∈S$
		\item $\|f*\hatμ\|_{L^q}\leq c^2\|f\|_{L^{q'}}\quad∀f∈S$
	\end{enumerate}
\end{lem}
Note, that $T:L^2→L^q,\ f↦\hat{fμ}$ (extension) iff $T^*:L^{q'}→L^2,\ g↦\hat g|_{\suppμ}$ (restriction) iff $TT*:L^{q'}→L^q,\ f↦f*\hatμ$

\begin{proof}[Proof of Tomas-Stein, up to endpoint]
Will show \[q>\f{2d+2}{d-1}⇒\|f*\hatμ\|_{L^q:ℝ^d)}\lesssim_{q,d}\|f\|_{L^{q'}(ℝ^d)}\]
Relevant properties of $σ$:
\begin{enumerate}
	\item $|\hatσ(ξ)|\lesssim(1+|ξ|)^{-\f{d-1}2}$
	\item $σ(D(x,r))\cong r^{d-1}$
\end{enumerate}
Note, that every measure satisfying this will have a Tomas-Stein, because these are the only properties we need.

Let $ϕ∈C^∞_0(ℝ^d),\ \supp(Φ)⊂\{x:\f14\leq|x|\leq 1\},\ \sum_{j\geq 0}φ(\f x{2^j})=1$ if $|x|\geq 1$.

Cur up $\hatσ$ as follows:
\[\hatσ=k_{-⟨}+\sum_{j=0}^∞k_j\]
with \[k_j(x)=ϕ(\f x{2^j})\hatσ(x)\]
and \[k_{-∞}(x)=(1-\sum_{j\geq 0}ϕ(\f x{2^j}))\hatσ(x)\]
Easy: $k_{-∞}∈C^∞_0$: \[\|f*k_{-∞}\|_{L^q}\lesssim\|f\|_{L^p}\quad∀p\leq q\]
Since $q>2$ we can take $p=q'$.

Trickier: $(k_j)_{j=0}^∞$. Upshot: Estimate the convolution with $k_j$ $L^1→L^∞,\ L^2→L^2$

$L^1→L^∞$.
\[\|f*k_j\|_{L^∞}\leq\|k_j\|_{L^∞}\|f\|_{L^1}\] where
\[\|k_j\|_{L^∞}\sim 2^{-j\f{d-1}2}\] as a consequence of property (i).

$L^2→L^2$
\[\|f*k_j\|_{L^2}=\|\hat f\hat k_j\|_{L^2}\leq\|\hat k_j\|_{L^∞}\|f\|_{L^2}\]
Claim :$\|\hat k_j\|_{L^∞}\sim 2^j$ (Next lecture)

Interpolate (Riesz-Thorin)
\[f*k_j\|_{L^q}\lesssim2^{-j\f{d-1}2(1-θ)}2^{jθ}\|f\|_{L^{q'}}\]
\[\f1q=\f{1-θ}∞+\fθ2\therefore q=\f2θ\]
which is equivalent to
\[\|f*k_j\|_{L^q}\lesssim 2^{j(\f{d+1}q-\f{d-1}2)}\|f\|_{L^{q'}}\]
for any $q∈[2,∞]$. The exponent is less than 0 iff $q>\f{2d+2}{d-1}$
\end{proof}
