\paragraph{Fourier restriction theory}
Given $f:\se^{d-1}→ℂ$, consider the Fourier transform $\wh{Fσ}(ξ)=∫_{\se^{d-1}}f(x)e^{-2πixξ}\mdσ(x)$ for $ξ∈ℝ^d$ (tempered distribution which turns out to be a function)
\begin{itemize}
	\item $f$ smooth $⇒\wh{fσ}$ decays at $∞$, e.g.
		\[|\wh{fσ}(ξ)|\leq c\|f\|_{C^2}(1+|ξ|)^{-\f{d-1}2}\]
		via stationary phase
	\item $f$ bounded then no pointwise decay holds in general. E.g.\ \[f_k(x)=e^{2πikx}\]
		Consider $ξ=k$ then $|\wh{f_kσ}(ξ)|=σ(\se^{d-1})\simeq 1$, no decay here. Now take
		\[f=\sum_{j\geq 0}\f{f_{k_j}}{j^2}\]
		where $|k_j|→∞$ sufficiently fast, e.g.\ $k_j=j!$. Easy to check: $f$ is continuous but $|\wh{fσ}(ξ)|\leq c(1+|ξ|)^{-ε}$ does not hold for any $c<∞,\ ε>0$. (uniformly in $ξ$).
	\item Problem of distinguished origin disappears if we take $L^q$ norms (There was some discussion on the wording going on?)
\end{itemize}
Restriction conjecture (Stein): Prove, that if $f∈L^∞(\se^{d-1})$, then
\[\|\wh{fσ}\|_{L^q(ℝ^d)}\leq c_{q,d}\|f\|_{L^∞(\se^{d-1},σ)}\]
for every $q>\f{2d}{d-1}$.
\begin{itemize}
	\item Range of exponents would be sharp: take $f\equiv 1∈L^∞(\se^{d-1})$
		\[\|\whσ\|_{L^q(ℝ^d)}^q=∫_{ℝ^d}|\whσ(ξ)|^q\md ξ\lesssim∫_{ℝ^d}(1+|ξ|)r^{-\f{d-1}2q}\mdξ\simeq_d∫_0^∞(1+r)^{-\f{d-1}2q}r^{d-1}\md r\]
		The last expression is $<∞$ iff $-\f{d-1}2q+(d-1)<-1$ iff $q>\f{2d}{d-1}$.

\end{itemize}
Corresponding problem for $L^2$ densities was solved by Tomas-Stein inequality ($\sim 1975$). If $f∈L^2(\se^{d-1})$, then
\[\|\wh{fσ}\|_{L^q(ℝ^d)}\lesssim_{q,d}\|f\|_{L^2(\se^{d-1})}\]
if $q\geq\f{2d+2}{d-1}$, and this range is the best possible. Note, that this was already proven last term and can be found in Stein/Shakarchi

\begin{itemize}
	\item Assumptions are of the form $q>q_0$ or $q\geq q_0$. Why?
		\[\|\wh{fσ}\|_{L^∞(ℝ^d)}=\sup_{ξ∈ℝ^d}|∫_{\se^{d-1}}f(x)e^{-2ηixξ}\md σ_x|\leq∫_{\se^{d-1}}|f(x)|\mdσ_x=\|f\|_{L^1(\se^{d-1})}\lesssim\|f\|_{L^2(\se^{d-1})}\]
		+ Riesz-Thorin because the sphere is compact
	\item $q\geq\f{2d+2}{d-1}$ is the best possible for $L^2$ densities: Knapp counterexample.
		\[C_δ=\{x∈\se^{d-1}:1-xe_d\leqδ^2\}\] where $e_d=(0,…,0,1)$. Since $|x-e_d|^2=2(1-xe_d)$, \[|x-e_d|\leq cδ⇒x∈C_δ⇒|x-e_d|\leq Cd\]
		for appropriate constants $0<c<C<∞$. Let $f=1_{C_δ}$. Right hand side is easy:
		\[\|f\|_{L^2(\se^{d-1})}=|C_δ|^{\f12}\simeq_dδq^{\f{d-1}2}\]
		Left hand is trickier. Note: the support of $fσ$ is contained in a cylindrical box $B_δ$ centered at $e_d$ with length $\simδ^2$ in the $_d$ direction and $\simδ$ in the $(d-1)$ orthogonal direction.
		\[\|\wh{fσ}\|_{L^q(ℝ^d)}\]
		Uncertainty principle: If a function is supported it a box, then its fourier transform is more or less constant on the dual box, which is the same but with inverse lengths. Idea: look at $\wh{fσ}$ on the dual box $B_δ^*$ (centered at $0$), u.e.\ suppose $|ξ_d|\leq c_1^{-1}δ^{-1}$ and $|ξ_j|\leq c_1^{-1}δ^{-1}$ if $j<d$, $c_1$ is a large constant to be chosen. If $ξ∈B_δ^*$, then
		\[|\wh{fσ}(ξ)|=|∫_{C_δ}e^{2πixξ}\mdσ_x|=∫e^{2πi(x-e_d)ξ}\mdσ_x|\geq∫_{C_δ}\cos(2π(x-e_d)ξ)\mdσ_x|\] Conditions on $ξ$ imply that 
		\[2π(x-e_d)ξ|\leq\fπ3\]
		if $C_1$ large enough. Therefore
		\[|\wh{fσ}(ξ)|\geq\f12|C_d|\simeqδ^{d-1}\]
		How large is $B_δ^*$?
		\[|B_δ^*|\simeqδ^{-2}(δ^{-1})^{d-1}=δ^{-(d+1)}\]
		Conclusion:
		\[\|\wh{fσ}\|_{L^1(ℝ^d)}\gtrsim(∫_{B_δ^*}|\wh{fσ}(ξ)|^q\mdξ)^{\f1q}\gtrsimδw{d-1}δ^{-\f{d+1}q}\]
		\[δ^{d-1}δ^{-\f{d+1}q}\leq\|\wh{fσ}\|_{L^q(ℝ^d)}\lesssim\|f\|_{L^2(\se^{d-1})}\simeqδ^{\f{d-1}2}\]
		\[\therefore d-1-\f{d+1}q\geq\f{d-1}2⇔\f{d-1}2\geq\f{d+1}q⇔q\geq{2d+2}{d-1}\]
\end{itemize}
So, if you violate both of the two obstructions, then the inequality should hold (which obstructions? Maybe $∞$-bound on the domain and on the range?))

Technical tool: convolution of Schwartz function with a (compactly supported) measure. $ϕ∈S(ℝ^d),\ μ∈M(ℝ^d)$ then
\[(φ*μ)(x)=∫φ(x-y)\mdμ(y)\]
Notation $\check μ=\whμ(-\cdot)$
\begin{lem}
	\begin{enumerate}
		\item \[\wh{\check ϕμ}=ϕ*\whμ\]
		\item \[\wh{ϕμ}=\whφ*\whμ\]
	\end{enumerate}
\end{lem}
\begin{proof}[Proof of (b), (a) is similar]
	Enough to show $∀ψ∈S(ℝ^d)$
	\[∫_{ℝ^d}\wh{ϕμ}ψ\md x=∫_{ℝ^d}(\whφ*\whμ)ψ\md x\]

	\begin{align*}
		∫_{ℝ^d}\wh{ϕμ}ψ\md x=∫\whψφ\md μ=∫\wh{\checkϕ*ψ}\mdμ=∫_{ℝ^d}(\checkφ*ψ)\whμ\md x=∫_{ℝ^d}(\whϕ*\whμ)ψ\md x
	\end{align*}
	due to duality, Fourier inversion on $S$, duality and definition of $*$ + Fubini.
\end{proof}
\begin{lem} $f,g∈S,μ∈M(ℝ^d)$. Then
	\[∫\wh f\bar{\wh g}\md μ=∫_{ℝ^d}(\whμ*\bar g)f\md x\]
\end{lem}
\begin{proof}
	\[∫\wh f\bar{\wh g}\md μ=∫_{ℝ^d}f\wh{\bar{\wh g}μ}\md x=∫_{ℝ^d}f(\bar g*\wh μ)\md x\]
	by duality and lemma 1
\end{proof}
\begin{lem} $μ$ finite positive measure. Then the following are equivalent
	\begin{enumerate}
		\item $\|\wh{fμ}\|_{L^q}\leq c\|f\|_{L^2(μ)}\quad∀f∈L^2(μ)$
		\item $\|\wh g\|_{L^2(μ)}\leq c\|g\|_{L^{q'}}\quad∀g∈S$
		\item $\|f*\whμ\|_{L^q}\leq c^2\|f\|_{L^{q'}}\quad∀f∈S$
	\end{enumerate}
\end{lem}
Note, that 
\begin{itemize}
	\item $T:L^2→L^q,\ f↦\wh{fμ}$ (extension) iff 
	\item $T^*:L^{q'}→L^2,\ g↦\wh g|_{\suppμ}$ (restriction) iff 
	\item $TT^*:L^{q'}→L^q,\ f↦f*\whμ$
\end{itemize}
Again, there were some words about the expression 'extension', which I did not understand.

\begin{proof}[Proof of Tomas-Stein, up to endpoint]
	Will show \[q>\f{2d+2}{d-1}⇒\|f*\whμ\|_{L^q(ℝ^d)}\lesssim_{q,d}\|f\|_{L^{q'}(ℝ^d)}\]
	Relevant properties of $σ$:
	\begin{enumerate}
		\item $|\whσ(ξ)|\lesssim(1+|ξ|)^{-\f{d-1}2}$
		\item $σ(D(x,r))\simeq r^{d-1}$
	\end{enumerate}
	Note, that every measure satisfying this will have a Tomas-Stein, because these are the only properties we need.

	Let $ϕ∈C^∞_0(ℝ^d),\ \supp(Φ)⊂\{x:\f14\leq|x|\leq 1\},\ \sum_{j\geq 0}φ(\f x{2^j})=1$ if $|x|\geq 1$.

	Cut up $\whσ$ as follows:
	\[\whσ=k_{-∞}+\sum_{j=0}^∞k_j\]
	with \[k_j(x)=ϕ(\f x{2^j})\whσ(x)\]
	and \[k_{-∞}(x)=(1-\sum_{j\geq 0}ϕ(\f x{2^j}))\whσ(x)\]
	Easy: $k_{-∞}∈C^∞_0$: \[\|f*k_{-∞}\|_{L^q}\lesssim\|f\|_{L^p}\quad∀p\leq q\]
	Since $q>2$ we can take $p=q'$ (Why does it hold for $q'$, then? And don't we want to show it only for $q'$ anyways?).

	Trickier: $(k_j)_{j=0}^∞$. Upshot: Estimate the convolution with $k_j$ $L^1→L^∞,\ L^2→L^2$.

	First $L^1→L^∞$.
	\[\|f*k_j\|_{L^∞}\leq\|k_j\|_{L^∞}\|f\|_{L^1}\] where
	\[\|k_j\|_{L^∞}\sim 2^{-j\f{d-1}2}\] as a consequence of property (i).

	$L^2→L^2$
	\[\|f*k_j\|_{L^2}=\|\wh f\wh k_j\|_{L^2}\leq\|\wh k_j\|_{L^∞}\|f\|_{L^2}\]
	Claim :$\|\wh k_j\|_{L^∞}\sim 2^j$ (Next lecture)

	Interpolate (Riesz-Thorin)
	\[\|f*k_j\|_{L^q}\lesssim2^{-j\f{d-1}2(1-θ)}2^{jθ}\|f\|_{L^{q'}}\]
	\[\f1q=\f{1-θ}∞+\fθ2\therefore q=\f2θ\]
	which is equivalent to
	\[\|f*k_j\|_{L^q}\lesssim 2^{j(\f{d+1}q-\f{d-1}2)}\|f\|_{L^{q'}}\]
	for any $q∈[2,∞]$. The exponent is less than 0 iff $q>\f{2d+2}{d-1}$
\end{proof}
